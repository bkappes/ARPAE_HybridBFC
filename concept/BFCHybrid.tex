\documentclass[12pt]{amsart}
\usepackage{geometry}                % See geometry.pdf to learn the layout options. There are lots.
\geometry{letterpaper, margin=1in}                   % ... or a4paper or a5paper or ... 
%\geometry{landscape}                % Activate for for rotated page geometry
%\usepackage[parfill]{parskip}    % Activate to begin paragraphs with an empty line rather than an indent
\usepackage{graphicx}
\usepackage{amssymb}
\usepackage{epstopdf}
\usepackage{fancyhdr}
%\usepackage[super,sort&compress]{natbib}
\usepackage[superscript,biblabel]{cite}
\usepackage{wrapfig}
\usepackage[small,bf]{caption}
\setlength{\captionmargin}{0pt}

\DeclareGraphicsRule{.tif}{png}{.png}{`convert #1 `dirname #1`/`basename #1 .tif`.png}

\pagestyle{fancy}
% page 1
\fancypagestyle{first style} {
	\fancyhf{} % clear all existing styles
	\lfoot{\scriptsize %
		\vspace{1em} %
		Contains Confidential, Proprietary, or Privileged Information \\ %
		Exempt from Public Disclosure \hfill \thepage \hfill Control \#1261-2649%
	}
}
% pages 2 and following
\lhead{\scriptsize %
	DE-FOA-0001261 \\ %
	Concept Paper %
}
\cfoot{} % suppress default center page number
\lfoot{\scriptsize %
	\vspace{1em}%
	Contains Confidential, Proprietary, or Privileged Information \\ %
	Exempt from Public Disclosure \hfill \thepage \hfill Control \#1261-2649%
}

\title[Hybrid DMFC/ASSLB]{Methanol Fuel Cell Enabled Hybrid Power System}
\author[Ban, Christensen, Ciobanu, Kappes, Hurst]{%
	Process Global, Inc. (Sunnyvale, CA); Dr. Branden Kappes \\ %
	Technical Category: 7.L Technologies for portable power applications \\ %
	Estimated Total Project Cost: \$3.6M \\ %
	Projection Duration: 3 years
}
%\date{}                                           % Activate to display a given date or no date

\begin{document}
\maketitle

\thispagestyle{first style}
\paragraph{\bf Summary} Next generation portable electronics will increasingly require always-on technologies -- such as powered sensors, continuous data acquisition, on-chip data analytics and communication -- to improve functionality and extend into applications currently prohibited by relatively high energy demands.  Lithium ion batteries are sufficient to power existing portable systems, and operate at or near 100\% Coulombic efficiency over  a wide range of currents and state of charge. But energy densities below 100 Wh/L place prohibitive limits on technologies with always-on energy requirements.  In contrast, direct methanol fuel cells (DMFCs) have theoretical energy densities in excess of 3000 Wh/L, but suffer from chemical-, mass transport-, and ohmic-polarization losses that impede their ability to promptly respond to changes in power demand, resulting in a narrow optimal operating range.

A hybrid battery-fuel cell power system couples the energy density and generation of a DMFC with the power response and wide operational efficiency of an all solid-state lithium ion battery (ASSLB). Figure~\ref{fig:efficiency} shows the potential impact of a hybrid DMFC/ASSLB power system.  A survey of coal, petroleum, natural gas, and nuclear fuels powering steam, gas turbine, ICE, and combined cycle generation reveal a 73\% fuel-to-use loss.  Presently, DMFCs operate at an efficiency of $\sim$19\%, but unconstrained by Carnot efficiency, these could theoretically achieve a 79\% efficiency, including losses from production and distribution. As a result, portable devices powered by our hybrid DMFC/ASSLB would require considerably less fuel than those powered off the electric grid.

\begin{wrapfigure}{l}{0.5\textwidth}
	\begin{center}
		\includegraphics[width=0.48\textwidth]{fig/efficiency}
		\caption{Potential for improved energy efficiency through use of the proposed hybrid direct methanol fuel cell/all solid-state lithium battery power system.}
		\label{fig:efficiency}
	\end{center}
\end{wrapfigure}
This project will, over three years, address the three challenges facing DMFC that keep their efficiencies below our 35\% target: high methanol crossover, high anode polarization due to low catalyst activity, and high cathode polarization due to mixed potential losses. In order to reduce battery complexity, and improve the volumetric energy density of the hybrid power system, this project will improve battery technology through the development of an all solid-state lithium ion battery.  The resulting hybrid DMFC/ASSLB will allow the power and energy requirements of each application to be optimized independently.

{\bf Proposed Work} Graphene oxide (GO) has recently been identified as a membrane with extremely high selectivity and permeability to water.\cite{Nair2012} The low intrinsic proton conductivity of GO\cite{Tateishi2013} can be improved through sulfonation to a level comparable to Nafion${}^\textrm{\textregistered}$,\cite{Scott2012a} but at the cost of increased methanol permeability.\cite{Jiang2014}  Furthermore, at high methanol concentrations the sulfonic acid groups on the sGO surface induce a methanol/water phase separation that reduces proton conductivity.\cite{Paneri2014}  The precise nature of proton transport through GO is not known, but the insensitivity of proton transport to GO flake size (Figure~\ref{fig:Fig7Paneri}) suggests through-platelet transport plays a dominant role; contrarily, the decrease in methanol permeability over that same range indicates methanol permeation occurs predominantly at platelet edges.

\begin{figure}[t]
	\begin{minipage}{0.45 \textwidth}
		\begin{center}
		\includegraphics[width=\textwidth]{fig/Fig7Paneri2014}
		\captionof{figure}{Methanol permeability (diffusivity) and proton conductivity through graphene oxide (GO) as a function of the mean nanoplatelet size. From Paneri, et al.\cite{Paneri2014}}
		\label{fig:Fig7Paneri}
		\end{center}
	\end{minipage} %
	\hfill %
	\begin{minipage}{0.45 \textwidth}
		\begin{center}
		\includegraphics[width=\textwidth]{fig/NPModeling}
		\captionof{figure}{(a) Computationally guided synthesis of metallic nanoparticles by wet chemical reduction (WCR) �recent work at CSM by Richards and Ciobanu.\cite{Leong2014} (b) Key parameters of the computational model: net attachment rates for each facet. (c) Predictions of the model for different ratios of attachement rates to (001) and (111) facets. (d) Actual Pd cuboctahedra synthesized by Richards' group.}
		\label{fig:NPModeling}
		\end{center}
	\end{minipage}
	\vspace{-1em}
\end{figure}

Limited work has been done on the chemical modification of GO for DMFC membranes. The proposed work addresses two of the three problems: proton conductivity (ohmic losses) and methanol crossover (mixed potential losses).  We propose to modify the GO surface using vapor phase methods, including ALD and molecular layer deposition (MLD), to enable the development of GO membranes that resist methanol/water phase separation while increasing proton transport.  Molecular dynamics simulations of proton transport through graphene oxide will be used to evaluate the efficacy of chemically modified GO membranes.  For DMFC, success would be an increase in the methanol concentration from 2 to 10 M; a decrease in methanol permeability from 50 to 0.25~$\mathrm{mA/cm^2}$; an increase in the proton conductivity for GO from 0.0045 to 0.05~S/cm; and a twofold reduction in price, from \$550/kWh to \$250/kWh.

Platinum group metal (PGM) catalyst activity is affected by both the nature of the substrate,\cite{Feng2013a} and by the shape and size of the catalyst nanoparticles (NPs).  The proposed effort will increase catalytic activity by optimization of the catalyst synthesis to produce nanoparticles of prescribed morphologies, shapes and sizes. We will pursue catalyst optimization by combining computational tools with synthesis and characterization to improve catalyst activity and achieve a fundamental understanding of how synthesis conditions affect the final product and its catalytic properties, c.f. Figure~\ref{fig:NPModeling}.

The catalytic materials systems to be addressed are Pt and Pt-alloy NPs, with or without core-shell morphologies.  These well-defined catalytic systems are ideal for linking experiments and modeling, providing controlled systems for building selective and complex functionalities.  The proposed effort will increase catalyst stability by lowering the solubility of Ru in acidic media. Our team has extensive experience in Pt--Ru deposition including wet chemical reduction (WCR), ALD, and sputtering.  In addition to technical challenges, catalyst cost poses a significant economic challenge.  Both anode and cathode require PGM catalysts.  Loading levels of 2.5~$\mathrm{mg/cm^2}$ account for a price of \$1366/kW (Pt: \$1162/oz).  To reduce catalytic loading, and bring the price below competing technologies, we will also investigate the targeted growth of Pt to GO surface defects using ALD, localizing catalyst deposition to proton transport centers.

Solid-state electrolytes appreciably reduce the complexity of each lithium ion cell, reducing both weight and volume.  With a smaller cell, this volume and weight can be recaptured into increased fuel storage, multiplying the impact of any improvement in battery performance.  An all solid-state battery using solid electrolytes is expected to have a higher energy density, reliability, and reduced safety concerns compared to a lithium ion battery using organic liquid electrolytes. All-solid-state batteries can be divided into two types, thin-film-type and bulk-type. For large-scale applications, bulk-type ASSLBs with high loadings of active material and solid electrolyte powders are well-suited because of their high energy density.  Sulfide type electrolytes were developed to improve conductivity over earlier solid electrolytes, e.g. LiPON. $\mathrm{Li_2S}$--$\mathrm{P_2S_5}$ and $\mathrm{Li_2S}$--$\mathrm{P_2S_5}$--$\mathrm{GeS_2}$ systems offer ion conductivity from $10^{-3}$ to $10^{-2}$~S/cm at room temperature, similar to liquid electrolytes; and have a high, 5~V decomposition potential.  With these electrolytes, the maximum resistance is observed at the cathode/sulfide electrolyte interfaces and presents the most pressing technical challenge: improved cathode--electrolyte contact.

{\it Year 1} will focus on the development of individual components with significant progress toward the stated performance metrics.  This will include (1) synthesis and characterization of candidate graphene oxide and chemically modified GO membranes. (2) Synthesis, modeling, and analysis of the stability, activity, and performance of the anode catalyst layer under conditions near and around those expected during fuel cell operation.  (3) We will use SEM, TEM, and cyclic voltammetry to understand the cathode/electrolyte interface, which has been identified as an issue of the greatest importance for the improvement in ASSLBs.  (4) Finally, system level modeling efforts will be put in place to pre-optimize operating conditions based on the evolving properties of the catalyst layer, membrane and battery properties.

{\it Year 2} will focus on attaining the target performance metrics for all individual components. Fuel cell components will be integrated into a test cell for controlled performance testing at the Energy Systems Integration Facility at NREL.  Solid-state battery components will be integrated into a coin cell configuration for electrochemical testing.

{\it Year 3} will optimize the performance of the integrated fuel cell and battery systems under simulated real-world operational variations. Merging the fuel cell and solid-state battery into a hybrid power system will be done at PGi.

Key technical risks for ALD and MLD modification of GO platelets are the ineffectiveness of functionalization chemistries.  The risk will be mitigated by our flexibility in creating and modeling a range of candidate chemistries.  To address risks in catalyst development, our team has explored functionalization of the carbon support with nitrogen to prevent aggregation in Pt--Ru--R (R = N, C, B) catalysts in prior work and the capabilities are available to integrate this process into catalyst optimization.  The interplay between component properties under various operating conditions is an unknown whose risk will be mitigated through systems level modeling of fuel cell, battery, and hybrid system. By combining battery technology with direct methanol fuel cells, advancement in this hybrid system does not hinge on improvement in any single technology, but rather benefits from every individual improvement: in battery capacity, catalyst activity, or membrane performance.

%\begin{center}\textbf{References}\end{center}
%\bibliographystyle{plain}
%\bibliography{bibliography}
\begin{center}
\begin{minipage}{0.85\textwidth}
\begingroup
\renewcommand{\section}[2]{}
\begin{thebibliography}{1}

\bibitem{Feng2013a}
Chen Feng, et al., %Taizo Takeuchi, Mohammad~Ali Abdelkareem, Takuya Tsujiguchi, and Nobuyoshi Nakagawa.
\newblock {\em Journal of Power Sources}  {\bf 242} (2013) 57--64

\bibitem{Jiang2014}
Zhongqing Jiang, et al., %Yilei Shi, Zhong-Jie Jiang, Xiaoning Tian, Lijuan Luo, and Weiheng Chen.
\newblock {\em Journal of Materials Chemistry A} {\bf 2} n.18 (2014) 6494

\bibitem{Leong2014}
G.~Jeremy Leong, et al., %Abbas Ebnonnasir, Maxwell~C. Schulze, Matthew~B Strand, Chilan Ngo, David Maloney, Sarah~L. Frisco, Huyen~N. Dinh, Bryan Pivovar, George~H. Gilmer, Suneel Kodambaka, Cristian~V. Ciobanu, and Ryan~M. Richards.
\newblock {\em Nanoscale} {\bf 6} (2014) 11364--11371

\bibitem{Nair2012}
R~R Nair, et al., %H~a Wu, P~N Jayaram, I~V Grigorieva, and a~K Geim.
\newblock {\em Science} {\bf 335} (2012) 442--444

\bibitem{Paneri2014}
Abhilash Paneri, et al., %Yunseon Heo, Gregory Ehlert, Anton Cottrill, Henry Sodano, Peter Pintauro, and Saeed Moghaddam.
\newblock {\em Journal of Membrane Science} {\bf 467} (2014) 217--225

\bibitem{Scott2012a}
Keith Scott,
\newblock {\em Chemical Communications} {\bf 48} n.45 (2012) 5584

\bibitem{Tateishi2013}
H.~Tateishi, et al., %K.~Hatakeyama, C.~Ogata, K.~Gezuhara, J.~Kuroda, a.~Funatsu, M.~Koinuma, T.~Taniguchi, S.~Hayami, and Y.~Matsumoto.
\newblock {\em Journal of the Electrochemical Society} {\bf 160} n.11 (2013) F1175--F1178

\end{thebibliography}
\endgroup
\end{minipage}
\end{center}

\vspace{1em}
\begin{center}\textbf{Team Organization and Capabilities}\end{center}

\noindent\textbf{Project Prime}: Process Global, Inc. (PGi) has established unique partnerships of highly experienced industry leaders representing a multidisciplinary range of disruptive technologies. The core team has over 100+ man-years of experience across a wide range of fields, including manufacturing, process development, equipment engineering, and renewable energy.

\textit{Principle Investigator}: Dr. Branden Kappes is the VP of Technology at PGi.  He will be responsible for systematically studying electrochemical properties of the solid-state Li-ion battery and characterization and testing of the proposed hybrid power system.

\noindent\textbf{Project Partner}: The National Renewable Energy Laboratory (NREL) has a strong research effort in fuel cell catalyst development, which includes metal alloy WCR, ALD, sputtering; physical and chemical modification of fuel cell catalyst support; synthesis of advanced carbon materials; electrochemical testing and cycling; physical characterization of materials; fuel cell assembly and testing. NREL is also equipped with an electrochemical laboratory uniquely capable of synthesizing and testing lithium ion systems from materials to coin cells.

\textit{Key Member}: Dr. Chunmei Ban is staff scientist in the Center of Chemistry and Materials Science at NREL. Dr. Ban's expertise in the study of nanostructured materials has been instrumental in successfully implementing DOE-funded projects including Nanostructured Metal Oxide and high-energy electrode materials for lithium-ion batteries.

\textit{Key Member}: Dr. Steven Christensen is an expert in catalyst characterization and will support process feedback for the optimization of the DMFC components.

\textit{Key Member}: Dr. Katherine Hurst is an expert in material synthesis and will perform catalyst synthesis and GO formation and modification.

\noindent\textbf{Project Partner}: Colorado School of Mines maintains a strong institutional focus on energy research, including high performance computing resources dedicated to solving energy-related challenges using modeling and simulation, and extensive research in fuel cells through the Colorado Fuel Cell Center.

\textit{Key Member}: Prof. Cristian Ciobanu is an expert in computational materials science focusing on structure, properties, and phenomena in nanomaterials, and will provide molecular dynamics simulations and process modeling expertise to the project.

All members of the team have collaborated on projects relevant to the proposed effort. Drs. Ban and Kappes have collaborated on organic flow batteries through the ARPA-E sponsored RANGE program and for three years on nanostructured materials for next-generation Li-ion batteries. Drs. Ciobanu and Kappes have collaborated extensively on simulation and modeling across a range of materials challenges, including research into lithium ion batteries.

\end{document}  